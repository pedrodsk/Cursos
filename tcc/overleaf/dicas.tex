\chapter{DICAS}
\label{chap:dicas}

%%%%%%%%%%%%%%%%%%%%%%%%%%%% CAPÍTULOS %%%%%%%%%%%%%%%%%%%%%%%%%%%% 

\section{Referenciar capítulos.}

\ref{chap:fundamentacaoTeorica}

\autoref{chap:fundamentacaoTeorica}

%%%%%%%%%%%%%%%%%%%%%%%%%%%% CAPÍTULOS %%%%%%%%%%%%%%%%%%%%%%%%%%%% 

%%%%%%%%%%%%%%%%%%%%%%%%%%%% FIGURAS %%%%%%%%%%%%%%%%%%%%%%%%%%%%%%

\section{Referenciar figuras.}

\autoref{fig:figura-exemplo1}

\section{Adicionar figuras.}

\begin{figure}[!htb]
    \centering
    \caption{Exemplo de Figura}
    \includegraphics[width=0.5\textwidth]{./dados/figuras/figura1}
    \fonte{\citeonline{IRL2014}}
    \label{fig:figura-exemplo1}
\end{figure}

%%%%%%%%%%%%%%%%%%%%%%%%%%%% CAPÍTULOS %%%%%%%%%%%%%%%%%%%%%%%%%%%%%%

%%%%%%%%%%%%%%%%%%%%%%%%%% Quadros e Tabelas %%%%%%%%%%%%%%%%%%%%%%%% 

\section{Referenciar quadros e tabelas.}

\autoref{qua:quadro-exemplo1}

\autoref{tab:tabela-exemplo1}

\section{Adicionar quadros e tabelas.}

\verb|\input{./dados/quadros/quadro1}|

\verb|\input{./dados/tabelas/tabela1}|

\section{Exemplos de quadros e tabelas.}

\begin{quadro}[H]
    \centering
    \caption{Exemplo de Quadro.\label{qua:quadro-exemplo1}}
    \begin{tabular}{|p{7cm}|p{7cm}|}
        \hline
        \textbf{BD Relacionais} & \textbf{BD Orientados a Objetos} \\
        \hline
        Os dados são passivos, ou seja, certas operações limitadas podem ser automaticamente acionadas quando os dados são usados. Os dados são ativos, ou seja, as solicitações fazem com que os objetos executem seus métodos. & Os processos que usam dados mudam constantemente. \\
        \hline
    \end{tabular}
    \fonte{\citeonline{Barbosa2004}}
\end{quadro}

\begin{table}[!htb]
    \centering
    \caption[Resultado dos testes]{Resultado dos testes.
    \label{tab:tabela-exemplo1}}
    \begin{tabular}{rrrrr}
        \toprule
            & Valores 1 & Valores 2 & Valores 3 & Valores 4 \\
        \midrule
            Caso 1 & 0,86 & 0,77 & 0,81 & 163 \\
            Caso 2 & 0,19 & 0,74 & 0,25 & 180 \\
            Caso 3 & 1,00 & 1,00 & 1,00 & 170 \\
        \bottomrule
    \end{tabular}
    \fonte{\citeonline{Barbosa2004}}
\end{table}

%%%%%%%%%%%%%%%%%%%%%%%%%% Quadros e Tabelas %%%%%%%%%%%%%%%%%%%%%%%% 

%%%%%%%%%%%%%%%%%%%%%%%%%%%%%% Equações %%%%%%%%%%%%%%%%%%%%%%%%%%%%%

\section{Referenciar equações.}

\autoref{eq:equacao-exemplo1}

\ref{eq:equacao-exemplo2}

\section{Exemplos de equações.}

\begin{equation}
    X(s) = \int\limits_{t = -\infty}^{\infty} x(t) \, \text{e}^{-st} \, dt
    \label{eq:equacao-exemplo1}
\end{equation}

\begin{equation}
    F(u, v) = \sum_{m = 0}^{M - 1} \sum_{n = 0}^{N - 1} f(m, n) \exp \left[ -j 2 \pi \left( \frac{u m}{M} + \frac{v n}{N} \right) \right]
    \label{eq:equacao-exemplo2}
\end{equation}

%%%%%%%%%%%%%%%%%%%%%%%%%%%%%% Equações %%%%%%%%%%%%%%%%%%%%%%%%%%%%%

%%%%%%%%%%%%%%%%%%%%%%%%%%%%% Algoritmos %%%%%%%%%%%%%%%%%%%%%%%%%%%%

\section{Adicionar algoritmos.}

\input{./dados/algoritmos/algoritmo1}

\section{Exemplos de algoritmos.}

\begin{algorithm}
    \caption{Exemplo de Algoritmo}
    \KwIn{o número $n$ de vértices a remover, grafo original $G(V, E)$}
    \KwOut{grafo reduzido $G'(V,E)$}
    $removidos \leftarrow 0$ \\
    \While {removidos $<$ n } {
        $v \leftarrow$ Random$(1, ..., k) \in V$ \\
            \For {$u \in adjacentes(v)$} {
                remove aresta (u, v)\\
                $removidos \leftarrow removidos + 1$\\
            }
            \If {há  componentes desconectados} {
                remove os componentes desconectados\\
            }
        }
\end{algorithm}

%%%%%%%%%%%%%%%%%%%%%%%%%%%%% Algoritmos %%%%%%%%%%%%%%%%%%%%%%%%%%%%

%%%%%%%%%%%%%%%%%%%%%%%%%%%%%%%% LISTAS %%%%%%%%%%%%%%%%%%%%%%%%%%%%%

\section{Exemplos de listas.}

\begin{itemize}
    \item item não numerado 1
    \item item não numerado 2
    \begin{itemize}
        \item subitem não numerado 1
        \item subitem não numerado 2
        \item subitem não numerado 3
    \end{itemize}
    \item item não numerado 3
\end{itemize}
\vspace{5mm}
\begin{enumerate}
    \item item numerado 1
    \item item numerado 2
    \begin{enumerate}
        \item subitem numerado 1
        \item subitem numerado 2
        \item subitem numerado 3
    \end{enumerate}
    \item item numerado 3
\end{enumerate}

%%%%%%%%%%%%%%%%%%%%%%%%%%%%%%%% LISTAS %%%%%%%%%%%%%%%%%%%%%%%%%%%%%

%%%%%%%%%%%%%%%%%%%%%%%%%%%%%% CITAÇÕES %%%%%%%%%%%%%%%%%%%%%%%%%%%%%

\section{Citações indiretas.}

\citeonline{Maturana2003}  <- Meio da frase.

\cite{Maturana2003} <- Final da frase.


\section{Citação direta.}

\subsection{Quatro linhas ou mais:}

\begin{citacao}
    Desse modo, opera-se uma ruptura decisiva entre a reflexividade filosófica, isto é a possibilidade do sujeito de pensar e de refletir, e a objetividade científica. Encontramo-nos num ponto em que o conhecimento científico está sem consciência. Sem consciência moral, sem consciência reflexiva e também subjetiva. Cada vez mais o desenvolvimento extraordinário do conhecimento científico vai tornar menos praticável a própria possibilidade de reflexão do sujeito sobre a sua pesquisa \cite[p.~28]{Silva2000}.
\end{citacao}

\subsection{Três linhas ou menos:}

A epistemologia baseada na biologia parte do princípio de que "assumo que não posso fazer referência a entidades independentes de mim para construir meu explicar" \cite[p.~35]{Maturana2003}.
\vspace{5mm}

A epistemologia baseada na biologia de \citeonline[p.~35]{Maturana2003} parte do princípio de que "assumo que não posso fazer referência a entidades independentes de mim para construir meu explicar".

\section{Outros exemplos de citações}:

\citeonline{Maturana2003} \ \ \  \verb|\citeonline{Maturana2003}|

\citeonline{Barbosa2004} \ \ \   \verb|\citeonline{Barbosa2004}|

\cite[p.~28]{Silva2000} \ \ \  \verb|\cite[p.~28]{Silva2000}|

\citeonline[p.~33]{Silva2000} \ \ \   \verb|\citeonline[p.~33]{v}|

\cite[p.~35]{Maturana2003} \ \ \   \verb|\cite[p.~35]{Maturana2003}|

\citeonline[p.~35]{Maturana2003} \ \ \   \verb|\citeonline[p.~35]{Maturana2003}|

\cite{Barbosa2004,Maturana2003} \ \ \   \verb|\cite{Barbosa2004,Maturana2003}|

%%%%%%%%%%%%%%%%%%%%%%%%%%%%%% CITAÇÕES %%%%%%%%%%%%%%%%%%%%%%%%%%%%%
